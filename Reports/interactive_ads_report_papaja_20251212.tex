% Options for packages loaded elsewhere
\PassOptionsToPackage{unicode}{hyperref}
\PassOptionsToPackage{hyphens}{url}
\documentclass[
  english,
  man,floatsintext]{apa6}
\usepackage{xcolor}
\usepackage{amsmath,amssymb}
\setcounter{secnumdepth}{-\maxdimen} % remove section numbering
\usepackage{iftex}
\ifPDFTeX
  \usepackage[T1]{fontenc}
  \usepackage[utf8]{inputenc}
  \usepackage{textcomp} % provide euro and other symbols
\else % if luatex or xetex
  \usepackage{unicode-math} % this also loads fontspec
  \defaultfontfeatures{Scale=MatchLowercase}
  \defaultfontfeatures[\rmfamily]{Ligatures=TeX,Scale=1}
\fi
\usepackage{lmodern}
\ifPDFTeX\else
  % xetex/luatex font selection
\fi
% Use upquote if available, for straight quotes in verbatim environments
\IfFileExists{upquote.sty}{\usepackage{upquote}}{}
\IfFileExists{microtype.sty}{% use microtype if available
  \usepackage[]{microtype}
  \UseMicrotypeSet[protrusion]{basicmath} % disable protrusion for tt fonts
}{}
\makeatletter
\@ifundefined{KOMAClassName}{% if non-KOMA class
  \IfFileExists{parskip.sty}{%
    \usepackage{parskip}
  }{% else
    \setlength{\parindent}{0pt}
    \setlength{\parskip}{6pt plus 2pt minus 1pt}}
}{% if KOMA class
  \KOMAoptions{parskip=half}}
\makeatother
% Make \paragraph and \subparagraph free-standing
\makeatletter
\ifx\paragraph\undefined\else
  \let\oldparagraph\paragraph
  \renewcommand{\paragraph}{
    \@ifstar
      \xxxParagraphStar
      \xxxParagraphNoStar
  }
  \newcommand{\xxxParagraphStar}[1]{\oldparagraph*{#1}\mbox{}}
  \newcommand{\xxxParagraphNoStar}[1]{\oldparagraph{#1}\mbox{}}
\fi
\ifx\subparagraph\undefined\else
  \let\oldsubparagraph\subparagraph
  \renewcommand{\subparagraph}{
    \@ifstar
      \xxxSubParagraphStar
      \xxxSubParagraphNoStar
  }
  \newcommand{\xxxSubParagraphStar}[1]{\oldsubparagraph*{#1}\mbox{}}
  \newcommand{\xxxSubParagraphNoStar}[1]{\oldsubparagraph{#1}\mbox{}}
\fi
\makeatother
\usepackage{graphicx}
\makeatletter
\newsavebox\pandoc@box
\newcommand*\pandocbounded[1]{% scales image to fit in text height/width
  \sbox\pandoc@box{#1}%
  \Gscale@div\@tempa{\textheight}{\dimexpr\ht\pandoc@box+\dp\pandoc@box\relax}%
  \Gscale@div\@tempb{\linewidth}{\wd\pandoc@box}%
  \ifdim\@tempb\p@<\@tempa\p@\let\@tempa\@tempb\fi% select the smaller of both
  \ifdim\@tempa\p@<\p@\scalebox{\@tempa}{\usebox\pandoc@box}%
  \else\usebox{\pandoc@box}%
  \fi%
}
% Set default figure placement to htbp
\def\fps@figure{htbp}
\makeatother
% definitions for citeproc citations
\NewDocumentCommand\citeproctext{}{}
\NewDocumentCommand\citeproc{mm}{%
  \begingroup\def\citeproctext{#2}\cite{#1}\endgroup}
\makeatletter
 % allow citations to break across lines
 \let\@cite@ofmt\@firstofone
 % avoid brackets around text for \cite:
 \def\@biblabel#1{}
 \def\@cite#1#2{{#1\if@tempswa , #2\fi}}
\makeatother
\newlength{\cslhangindent}
\setlength{\cslhangindent}{1.5em}
\newlength{\csllabelwidth}
\setlength{\csllabelwidth}{3em}
\newenvironment{CSLReferences}[2] % #1 hanging-indent, #2 entry-spacing
 {\begin{list}{}{%
  \setlength{\itemindent}{0pt}
  \setlength{\leftmargin}{0pt}
  \setlength{\parsep}{0pt}
  % turn on hanging indent if param 1 is 1
  \ifodd #1
   \setlength{\leftmargin}{\cslhangindent}
   \setlength{\itemindent}{-1\cslhangindent}
  \fi
  % set entry spacing
  \setlength{\itemsep}{#2\baselineskip}}}
 {\end{list}}
\usepackage{calc}
\newcommand{\CSLBlock}[1]{\hfill\break\parbox[t]{\linewidth}{\strut\ignorespaces#1\strut}}
\newcommand{\CSLLeftMargin}[1]{\parbox[t]{\csllabelwidth}{\strut#1\strut}}
\newcommand{\CSLRightInline}[1]{\parbox[t]{\linewidth - \csllabelwidth}{\strut#1\strut}}
\newcommand{\CSLIndent}[1]{\hspace{\cslhangindent}#1}
\ifLuaTeX
\usepackage[bidi=basic]{babel}
\else
\usepackage[bidi=default]{babel}
\fi
% get rid of language-specific shorthands (see #6817):
\let\LanguageShortHands\languageshorthands
\def\languageshorthands#1{}
\ifLuaTeX
  \usepackage[english]{selnolig} % disable illegal ligatures
\fi
\setlength{\emergencystretch}{3em} % prevent overfull lines
\providecommand{\tightlist}{%
  \setlength{\itemsep}{0pt}\setlength{\parskip}{0pt}}
% Manuscript styling
\usepackage{upgreek}
\captionsetup{font=singlespacing,justification=justified}

% Table formatting
\usepackage{longtable}
\usepackage{lscape}
% \usepackage[counterclockwise]{rotating}   % Landscape page setup for large tables
\usepackage{multirow}		% Table styling
\usepackage{tabularx}		% Control Column width
\usepackage[flushleft]{threeparttable}	% Allows for three part tables with a specified notes section
\usepackage{threeparttablex}            % Lets threeparttable work with longtable

% Create new environments so endfloat can handle them
% \newenvironment{ltable}
%   {\begin{landscape}\centering\begin{threeparttable}}
%   {\end{threeparttable}\end{landscape}}
\newenvironment{lltable}{\begin{landscape}\centering\begin{ThreePartTable}}{\end{ThreePartTable}\end{landscape}}

% Enables adjusting longtable caption width to table width
% Solution found at http://golatex.de/longtable-mit-caption-so-breit-wie-die-tabelle-t15767.html
\makeatletter
\newcommand\LastLTentrywidth{1em}
\newlength\longtablewidth
\setlength{\longtablewidth}{1in}
\newcommand{\getlongtablewidth}{\begingroup \ifcsname LT@\roman{LT@tables}\endcsname \global\longtablewidth=0pt \renewcommand{\LT@entry}[2]{\global\advance\longtablewidth by ##2\relax\gdef\LastLTentrywidth{##2}}\@nameuse{LT@\roman{LT@tables}} \fi \endgroup}

% \setlength{\parindent}{0.5in}
% \setlength{\parskip}{0pt plus 0pt minus 0pt}

% Overwrite redefinition of paragraph and subparagraph by the default LaTeX template
% See https://github.com/crsh/papaja/issues/292
\makeatletter
\renewcommand{\paragraph}{\@startsection{paragraph}{4}{\parindent}%
  {0\baselineskip \@plus 0.2ex \@minus 0.2ex}%
  {-1em}%
  {\normalfont\normalsize\bfseries\itshape\typesectitle}}

\renewcommand{\subparagraph}[1]{\@startsection{subparagraph}{5}{1em}%
  {0\baselineskip \@plus 0.2ex \@minus 0.2ex}%
  {-\z@\relax}%
  {\normalfont\normalsize\itshape\hspace{\parindent}{#1}\textit{\addperi}}{\relax}}
\makeatother

\makeatletter
\usepackage{etoolbox}
\patchcmd{\maketitle}
  {\section{\normalfont\normalsize\abstractname}}
  {\section*{\normalfont\normalsize\abstractname}}
  {}{\typeout{Failed to patch abstract.}}
\patchcmd{\maketitle}
  {\section{\protect\normalfont{\@title}}}
  {\section*{\protect\normalfont{\@title}}}
  {}{\typeout{Failed to patch title.}}
\makeatother

\usepackage{xpatch}
\makeatletter
\xapptocmd\appendix
  {\xapptocmd\section
    {\addcontentsline{toc}{section}{\appendixname\ifoneappendix\else~\theappendix\fi: #1}}
    {}{\InnerPatchFailed}%
  }
{}{\PatchFailed}
\makeatother
\keywords{interaction design, advertising, donation, behavior\newline\indent Word count: 1190}
\usepackage{csquotes}
\usepackage{bookmark}
\IfFileExists{xurl.sty}{\usepackage{xurl}}{} % add URL line breaks if available
\urlstyle{same}
\hypersetup{
  pdftitle={Effect of Interactivity on Donation Intention: Mediated Roles of Playfulness, Social Presence, Sympathy, and Perceived Response Efficacy},
  pdfauthor={Zhanming Chen1},
  pdflang={en-EN},
  pdfkeywords={interaction design, advertising, donation, behavior},
  hidelinks,
  pdfcreator={LaTeX via pandoc}}

\title{Effect of Interactivity on Donation Intention: Mediated Roles of Playfulness, Social Presence, Sympathy, and Perceived Response Efficacy}
\author{Zhanming Chen\textsuperscript{1}}
\date{}


\shorttitle{Effect of Interactivity on Donation Intention}

\authornote{

I want to thank Professor Moin Syed for his guidance and hands-on lectures, as well as all the students in PSY 8960 for their help and feedback.

The authors made the following contributions. Zhanming Chen: Conceptualization, Methodology, Project administration, Visualization, Writing - Original Draft, Writing - Review \& Editing.

Correspondence concerning this article should be addressed to Zhanming Chen, Department of Design Innovation, University of Minnesota, 89 Church Street SE, Minneapolis, MN 55455, USA. E-mail: \href{mailto:chen8475@umn.edu}{\nolinkurl{chen8475@umn.edu}}

}

\affiliation{\vspace{0.5cm}\textsuperscript{1} University of Minnesota}

\abstract{%
Interactive advertising can improve consumers' purchase intentions, but its effectiveness in promoting donation behaviors is less explored. This study investigates the relationship between perceived interactivity and donation intention. Data were collected from 600 MTurk participants through a between-subject experimental design. Our findings indicate that perceived interactivity positively affects donation intention. Furthermore, this influence is mediated by perceived playfulness, sympathy, social presence, and perceived response efficacy.
}



\begin{document}
\maketitle

As one of the emerging forms of advertising, interactive advertising adds features to traditional advertising that allow consumers to interact with and participate in the ads (Bezjian-Avery, Calder, \& Iacobucci, 1998). These interactive features provide active control, two-way communication, and a sense of synchronous reaction for the audience during interaction (Liu \& Shrum, 2002; Lombard \& Snyder-Duch, 2001; McMillan \& Hwang, 2002). Traditional advertising, such as static image ads in shopping malls, provided a one-way channel for delivering marketing information to the audience in a controlled, linear narrative. Compared to traditional advertising, interactive advertising has been shown to increase both consumers' attitudes and purchase intentions (Macias, 2003) and further improve overall marketing performance (Gu et al., 2022). An example of interactive advertising was the 2015 augmented-reality-based ad launched by Coca-Cola in the United States, titled ``Next time you're thirsty, drink an ad'' (Dipdrop Branding Solution, 2015). During basketball games, viewers could enjoy a visual experience of a smooth transition from TV to mobile devices, as they watched Coke Zero being poured from a bottle on the TV screen into a glass in the mobile app. Coca-Cola then offered a coupon code redeemable for a bottle of Coke Zero (The Coca-Cola Company, 2015).

While previous research on interactive advertising has focused on its effects on consumption behaviors (e.g., Ahn, Ellie Jin, \& Seo, 2024), the impact of interactive advertising on donation intentions and behaviors has been minimally explored. In this study, we aim to understand how interactive advertisements influence donation attitudes and behaviors. Therefore, our research question is: how does the level of advertising interactivity impact the audience's donation intentions?

\subsection{The Present Study}\label{the-present-study}

Research has shown that interactive design features of advertisements (e.g., buttons and body gestures) contribute to higher purchase intention (Ahn et al., 2024; Gu et al., 2022). At the same time, the perceived interactivity of visual materials can positively influence the audience's emotions, such as perceived playfulness, levels of sympathy, and social presence (Hand \& Varan, 2009; Ide et al., 2021; Kang, Shin, \& Ponto, 2020). Furthermore, the audience's positive emotions, including perceived playfulness, sympathy, and social presence, were examined separately and shown to affect perceptions of efficacy and behavioral intentions (Chen, Dai, Yao, \& Li, 2019; Kim \& Yu, 2015). Additionally, researchers identified perceived response efficacy, the belief that consumers' help can make a meaningful difference for beneficiaries (Septianto \& Paramita, 2021; Sharma \& Morwitz, 2016), as an important factor in explaining donation behaviors.

Building on the prior research, we propose the following five hypotheses (H1--H5):

\begin{itemize}
\tightlist
\item
  H1. Perceived interactivity of advertisements will be positively related to (a) perceived playfulness, (b) level of sympathy, and (c) social presence.
\item
  H2. Perceived playfulness will be positively related to (a) perceived response efficacy and (b) donation intention.
\item
  H3. Level of sympathy will be positively related to (a) perceived response efficacy and (b) donation intention.
\item
  H4. Social presence will be positively related to (a) perceived response efficacy and (b) donation intention.
\item
  H5. Perceived efficacy will be positively related to donation intention.
\end{itemize}

\section{Method}\label{method}

We conducted a between-subjects online experiment with two groups: high-interactivity and low-interactivity (controlled group). The stimulus is designed as an ad that encourages participants to donate winter clothing to a local homeless girl. For interactive features, participants in two groups either used a drag-and-drop feature to ``donate'' three pieces of clothing (i.e., high-interactivity) or watched an animation of the giving process for those three pieces of clothing (i.e., low-interactivity).

The current study was \textbf{NOT} preregistered. Simulated data and codes are available at \href{https://osf.io/ept28/files}{OSF} and \href{https://github.com/zhanmingchen/PSY8960-Fall25-Zhanming}{GitHub}.

\subsection{Participants and Procedure}\label{participants-and-procedure}

The total sample in the current study consists of 600 US adults recruited from Amazon MTurk (\(M_{\text{age}}\) = 38.85, \(SD_{\text{age}}\) = 8.01). They were evenly assigned to one of the two conditions. The experiment begins with a consent process, followed by participants interacting with the stimuli, and then an online questionnaire including measures of variables and demographic questions. The questionnaire also includes measures of graphic qualities to ensure differences in these qualities between the two groups are not significant, and three focus questions to ensure participants answer the questions with their focus.

The eligibility criteria of participants include being adults, living in the US, being comfortable speaking English, and having access to a laptop and the Internet to complete the tasks and survey using a web browser. Each participant is compensated with a \$25 Amazon gift card.

Here is a table of the detailed breakdown of participants' demographic information:

\begin{center}
\begin{ThreePartTable}

\begin{longtable}{llll}\noalign{\getlongtablewidth\global\LTcapwidth=\longtablewidth}
\caption{\label{tab:demographics table}Demographic Breakdowns of Participants}\\
\toprule
Item & \multicolumn{1}{c}{Category} & \multicolumn{1}{c}{Frequency} & \multicolumn{1}{c}{Percentage}\\
\midrule
\endfirsthead
\caption*{\normalfont{Table \ref{tab:demographics table} continued}}\\
\toprule
Item & \multicolumn{1}{c}{Category} & \multicolumn{1}{c}{Frequency} & \multicolumn{1}{c}{Percentage}\\
\midrule
\endhead
Gender & Female & 191 & 31.83\%\\
 & Male & 163 & 27.17\%\\
 & Non-binary or third gender & 187 & 31.17\%\\
Race & White & 90 & 15\%\\
 & Black or African American & 93 & 15.5\%\\
 & American Indian or Alaska Native & 103 & 17.17\%\\
 & Asian & 109 & 18.17\%\\
 & Native Hawaiian or Pacific Islander & 88 & 14.67\%\\
Income & Less than \$10,000 & 40 & 6.67\%\\
 & \$10,000 - \$19,999 & 56 & 9.33\%\\
 & \$20,000 - \$29,999 & 44 & 7.33\%\\
 & \$30,000 - \$39,999 & 50 & 8.33\%\\
 & \$40,000 - \$49,999 & 47 & 7.83\%\\
 & \$50,000 - \$59,999 & 42 & 7\%\\
 & \$60,000 - \$69,999 & 44 & 7.33\%\\
 & \$70,000 - \$79,999 & 56 & 9.33\%\\
 & \$80,000 - \$89,999 & 52 & 8.67\%\\
 & \$90,000 - \$99,999 & 51 & 8.5\%\\
 & \$100,000 - \$149,999 & 51 & 8.5\%\\
 & More than \$150,000 & 43 & 7.17\%\\
Education & Less than high school & 80 & 13.33\%\\
 & High school graduate & 84 & 14\%\\
 & 2 year degree & 88 & 14.67\%\\
 & 4 year degree & 84 & 14\%\\
 & Professional degree & 66 & 11\%\\
 & Doctorate & 83 & 13.83\%\\
\bottomrule
\end{longtable}

\end{ThreePartTable}
\end{center}

\subsection{Measures}\label{measures}

\subsubsection{Perceived Graphics Quality.}\label{perceived-graphics-quality.}

As a manipulation check, participants completed a three-item, 7-point Likert scale (Kang et al., 2020), which assessed their perceived graphics quality (Cronbach's alpha = -0.02).

\subsubsection{Perceived Interactivity.}\label{perceived-interactivity.}

Participants completed a five-item, 7-point Likert scale (Wu, 2005; Yim, Chu, \& Sauer, 2017), which assessed their perceived interactivity (Cronbach's alpha = 0.85).

\subsubsection{Social Presence.}\label{social-presence.}

Participants completed a ten-item, 7-point Likert scale (Higgins, Zibrek, Cabral, Egan, \& McDonnell, 2022), which assessed their social presence (Cronbach's alpha = 0.84).

\subsubsection{Sympathy.}\label{sympathy.}

Participants completed a 10-item, 7-point Likert scale (Baberini, Coleman, Slovic, \& Västfjäll, 2015), which assessed their sympathy (Cronbach's alpha = 0.91).

\subsubsection{Perceived Playfulness.}\label{perceived-playfulness.}

Participants completed a four-item, 7-point Likert scale (Kang et al., 2020), which assessed their perceived playfulness (Cronbach's alpha = 0.81).

\subsubsection{Perceived Response Efficacy.}\label{perceived-response-efficacy.}

Participants completed a four-item, 7-point Likert scale (Cryder, Loewenstein, \& Scheines, 2013; Sharma \& Morwitz, 2016), which assessed their perceived response efficacy (Cronbach's alpha = 0.81).

\subsubsection{Donation Intention.}\label{donation-intention.}

Participants completed a three-item, 7-point Likert scale (Li \& Yin, 2022), which assessed their donation intention (Cronbach's alpha = 0.81).

\subsection{Data analysis}\label{data-analysis}

We used R (Version 4.5.1; R Core Team, 2024) and the R-packages \emph{apaTables} (Stanley, 2021), \emph{dplyr} (Version 1.1.4; Wickham, François, Henry, Müller, \& Vaughan, 2023), \emph{faux} (Version 1.2.3; DeBruine, 2025), \emph{ggplot2} (Version 4.0.0; Wickham, 2016), \emph{groundhog} (Version 3.2.3; Simonsohn \& Gruson, 2025), \emph{knitr} (Version 1.50; Xie, 2015), \emph{labelled} (Version 2.16.0; Larmarange, 2025), \emph{missMethods} (Version 0.4.0; Rockel, 2022), \emph{papaja} (Version 0.1.4; Aust \& Barth, 2025), \emph{psych} (Version 2.5.6; William Revelle, 2025), \emph{summarytools} (Version 1.1.4; Comtois, 2025), and \emph{tinylabels} (Version 0.2.5; Barth, 2025) for all our analyses.

\section{Results}\label{results}

\subsection{Manipulation Tests}\label{manipulation-tests}

Overall, the graphic quality of the stimuli for two participant groups is well perceived (\(M_{\text{graphic quality}}\) = 5.68, \(SD_{\text{graphic quality}}\) = 0.89). The results from a Welch's independent samples \emph{t}-test indicated no significant differences between the two groups, \emph{t}(597.16) = -1.43, \emph{p} = .153 (see Figure \ref{fig:graphi-graph}).

\begin{figure}
\centering
\pandocbounded{\includegraphics[keepaspectratio]{interactive_ads_report_papaja_20251212_files/figure-latex/graphi-graph-1.pdf}}
\caption{\label{fig:graphi-graph}Perceived graphic quality of stimuli.}
\end{figure}

According to the results from a Welch's independent samples \emph{t}-test that compares the perceived interactivity of the stimuli between the high-interactivity group (\(M_{\text{high interactivity}}\) = 5.66, \(SD_{\text{high interactivity}}\) = 0.71) and the low-interactivity group (\(M_{\text{low interactivity}}\) = 2.36, \(SD_{\text{low interactivity}}\) = 0.78), there are significant differences between the two groups, \emph{t}(592.02) = 54.40, \emph{p} \textless{} .001 (see Figure \ref{fig:interactivity-graph}).

\begin{figure}
\centering
\pandocbounded{\includegraphics[keepaspectratio]{interactive_ads_report_papaja_20251212_files/figure-latex/interactivity-graph-1.pdf}}
\caption{\label{fig:interactivity-graph}Perceived interactivity of stimuli.}
\end{figure}

\subsection{Hypotheses Validation}\label{hypotheses-validation}

A series of Pearson correlation tests was conducted to assess the hypotheses. All hypotheses were supported (see the table below).

\begin{center}
\begin{ThreePartTable}

\begin{longtable}{llll}\noalign{\getlongtablewidth\global\LTcapwidth=\longtablewidth}
\caption{\label{tab:hypothesis-table}Hypotheses Test Results}\\
\toprule
Hypothesis & Pairs & \textit{t} & \textit{p}\\
\midrule
\endfirsthead
\caption*{\normalfont{Table \ref{tab:hypothesis-table} continued}}\\
\toprule
Hypothesis & Pairs & \textit{t} & \textit{p}\\
\midrule
\endhead
H1a & Interactivity $\leftrightarrow$ Playfulness & 34.35 & < .001\\
H1b & Interactivity $\leftrightarrow$ Sympathy & 43.98 & < .001\\
H1c & Interactivity $\leftrightarrow$ Social presence & 37.32 & < .001\\
H2a & Playfulness $\leftrightarrow$ Response efficacy & 33.10 & < .001\\
H2b & Playfulness $\leftrightarrow$ Donation intention & 32.43 & < .001\\
H3a & Sympathy $\leftrightarrow$ Response efficacy & 40.76 & < .001\\
H3b & Sympathy $\leftrightarrow$ Donation intention & 40.80 & < .001\\
H4a & Social presence $\leftrightarrow$ Response efficacy & 34.20 & < .001\\
H4b & Social presence $\leftrightarrow$ Donation intention & 35.40 & < .001\\
H5 & Response efficacy $\leftrightarrow$ Donation intention & 34.12 & < .001\\
\bottomrule
\end{longtable}

\end{ThreePartTable}
\end{center}

\section{Discussion}\label{discussion}

This study aimed to explore how perceived interactivity influences donation intention in advertisement design. The results supported our hypotheses, showing that this relationship is mediated by perceived playfulness, sympathy, social presence, and response efficacy. As a result, these findings suggest that advertisers and marketers should focus on interactive features to effectively encourage donation intentions. Future research should explore which design elements can effectively improve perceived interactivity.

\newpage

\section{References}\label{references}

\phantomsection\label{refs}
\begin{CSLReferences}{1}{0}
\bibitem[\citeproctext]{ref-ahn2024}
Ahn, S., Ellie Jin, B., \& Seo, H. (2024). Why do people interact and buy in the Metaverse? Self-Expansion perspectives and the impact of hedonic adaptation. \emph{Journal of Business Research}, \emph{175}, 114557. \url{https://doi.org/10.1016/j.jbusres.2024.114557}

\bibitem[\citeproctext]{ref-R-papaja}
Aust, F., \& Barth, M. (2025). \emph{{papaja}: {Prepare} reproducible {APA} journal articles with {R Markdown}}. \url{https://doi.org/10.32614/CRAN.package.papaja}

\bibitem[\citeproctext]{ref-baberini2015}
Baberini, M., Coleman, C.-L., Slovic, P., \& Västfjäll, D. (2015). Examining the Effects of Photographic Attributes on Sympathy, Emotions, and Donation Behavior. \emph{Visual Communication Quarterly}, \emph{22}(2), 118--128. \url{https://doi.org/10.1080/15551393.2015.1061433}

\bibitem[\citeproctext]{ref-R-tinylabels}
Barth, M. (2025). \emph{{tinylabels}: Lightweight variable labels}. \url{https://doi.org/10.32614/CRAN.package.tinylabels}

\bibitem[\citeproctext]{ref-bezjian-avery1998}
Bezjian-Avery, A., Calder, B., \& Iacobucci, D. (1998). New media interactive advertising vs. traditional advertising. \emph{Journal of Advertising Research}, \emph{38}(4), 23--32. Retrieved from \url{http://www.scopus.com/inward/record.url?scp=0002101166&partnerID=8YFLogxK}

\bibitem[\citeproctext]{ref-chen2019}
Chen, Y., Dai, R., Yao, J., \& Li, Y. (2019). Donate Time or Money? The Determinants of Donation Intention in Online Crowdfunding. \emph{Sustainability}, \emph{11}(16). \url{https://doi.org/10.3390/su11164269}

\bibitem[\citeproctext]{ref-R-summarytools}
Comtois, D. (2025). \emph{Summarytools: Tools to quickly and neatly summarize data}. \url{https://doi.org/10.32614/CRAN.package.summarytools}

\bibitem[\citeproctext]{ref-cryder2013}
Cryder, C. E., Loewenstein, G., \& Scheines, R. (2013). The donor is in the details. \emph{Organizational Behavior and Human Decision Processes}, \emph{120}(1), 15--23. \url{https://doi.org/10.1016/j.obhdp.2012.08.002}

\bibitem[\citeproctext]{ref-R-faux}
DeBruine, L. (2025). \emph{Faux: Simulation for factorial designs}. Zenodo. \url{https://doi.org/10.5281/zenodo.2669586}

\bibitem[\citeproctext]{ref-dipdropbrandingsolution2015}
Dipdrop Branding Solution. (2015). \emph{Coca cola Creates First Ever Drinkable Advertising Campaign}. Retrieved from \url{https://www.youtube.com/watch?v=IQovoot_ZUM}

\bibitem[\citeproctext]{ref-gu2022}
Gu, C., Lin, S., Sun, J., Yang, C., Chen, J., Jiang, Q., \ldots{} Wei, W. (2022). What do users care about? Research on user behavior of mobile interactive video advertising. \emph{Heliyon}, \emph{8}(10), e10910. \url{https://doi.org/10.1016/j.heliyon.2022.e10910}

\bibitem[\citeproctext]{ref-hand2009}
Hand, S., \& Varan, D. (2009). \emph{Interactive stories and the audience: Why empathy is important}. \emph{7}(3). \url{https://doi.org/10.1145/1594943.1594951}

\bibitem[\citeproctext]{ref-higgins2022}
Higgins, D., Zibrek, K., Cabral, J., Egan, D., \& McDonnell, R. (2022). Sympathy for the digital: Influence of synthetic voice on affinity, social presence and empathy for photorealistic virtual humans. \emph{Computers \& Graphics}, \emph{104}, 116--128. \url{https://doi.org/10.1016/j.cag.2022.03.009}

\bibitem[\citeproctext]{ref-ide2021}
Ide, M., Oshima, S., Mori, S., Yoshimi, M., Ichino, J., \& Tano, S. (2021). \emph{Effects of Avatar{'}s Symbolic Gesture in Virtual Reality Brainstorming}. 170177. New York, NY, USA: Association for Computing Machinery. \url{https://doi.org/10.1145/3441000.3441081}

\bibitem[\citeproctext]{ref-kang2020}
Kang, H. J., Shin, J., \& Ponto, K. (2020). How 3D Virtual Reality Stores Can Shape Consumer Purchase Decisions: The Roles of Informativeness and Playfulness. \emph{Journal of Interactive Marketing}, \emph{49}(1), 70--85. \url{https://doi.org/10.1016/j.intmar.2019.07.002}

\bibitem[\citeproctext]{ref-kim2015}
Kim, N., \& Yu, S. Y. (2015). Effect of the characteristics of models of public service advertisements on public service behavior intension: Mediated effect on attitude of PSA. \emph{Indian Journal of Science and Technology}, \emph{8}(S8), 250257. \url{https://doi.org/10.17485/ijst/2015/v8iS8/70526}

\bibitem[\citeproctext]{ref-R-labelled}
Larmarange, J. (2025). \emph{Labelled: Manipulating labelled data}. Retrieved from \url{https://CRAN.R-project.org/package=labelled}

\bibitem[\citeproctext]{ref-li2022}
Li, M.-R., \& Yin, C.-Y. (2022). Facial expressions of beneficiaries and donation intentions of potential donors: Effects of the number of beneficiaries in charity advertising. \emph{Journal of Retailing and Consumer Services}, \emph{66}, 102915. \url{https://doi.org/10.1016/j.jretconser.2022.102915}

\bibitem[\citeproctext]{ref-liu2002}
Liu, Y., \& Shrum, L. J. (2002). What is Interactivity and is it Always Such a Good Thing? Implications of Definition, Person, and Situation for the Influence of Interactivity on Advertising Effectiveness. \emph{Journal of Advertising}, \emph{31}(4), 53--64. \url{https://doi.org/10.1080/00913367.2002.10673685}

\bibitem[\citeproctext]{ref-lombard2001}
Lombard, M., \& Snyder-Duch, J. (2001). Interactive Advertising and Presence. \emph{Journal of Interactive Advertising}, \emph{1}(2), 56--65. \url{https://doi.org/10.1080/15252019.2001.10722051}

\bibitem[\citeproctext]{ref-macias2003}
Macias, W. (2003). A Preliminary Structural Equation Model of Comprehension and Persuasion of Interactive Advertising Brand Web Sites. \emph{Journal of Interactive Advertising}, \emph{3}(2), 36--48. \url{https://doi.org/10.1080/15252019.2003.10722072}

\bibitem[\citeproctext]{ref-mcmillan2002}
McMillan, S. J., \& Hwang, J.-S. (2002). Measures of Perceived Interactivity: An Exploration of the Role of Direction of Communication, User Control, and Time in Shaping Perceptions of Interactivity. \emph{Journal of Advertising}, \emph{31}(3), 29--42. \url{https://doi.org/10.1080/00913367.2002.10673674}

\bibitem[\citeproctext]{ref-R-base}
R Core Team. (2024). \emph{R: A language and environment for statistical computing}. Vienna, Austria: R Foundation for Statistical Computing. Retrieved from \url{https://www.R-project.org/}

\bibitem[\citeproctext]{ref-R-missMethods}
Rockel, T. (2022). \emph{missMethods: Methods for missing data}. \url{https://doi.org/10.32614/CRAN.package.missMethods}

\bibitem[\citeproctext]{ref-septianto2021}
Septianto, F., \& Paramita, W. (2021). Sad but smiling? How the combination of happy victim images and sad message appeals increase prosocial behavior. \emph{Marketing Letters}, \emph{32}(1), 91--110. \url{https://doi.org/10.1007/s11002-020-09553-5}

\bibitem[\citeproctext]{ref-sharma2016}
Sharma, E., \& Morwitz, V. G. (2016). Saving the masses: The impact of perceived efficacy on charitable giving to single vs. multiple beneficiaries. \emph{Organizational Behavior and Human Decision Processes}, \emph{135}, 45--54. \url{https://doi.org/10.1016/j.obhdp.2016.06.001}

\bibitem[\citeproctext]{ref-R-groundhog}
Simonsohn, U., \& Gruson, H. (2025). \emph{Groundhog: Version-control for CRAN, GitHub, and GitLab packages}. Retrieved from \url{https://CRAN.R-project.org/package=groundhog}

\bibitem[\citeproctext]{ref-R-apaTables}
Stanley, D. (2021). \emph{apaTables: Create american psychological association (APA) style tables}. Retrieved from \url{https://CRAN.R-project.org/package=apaTables}

\bibitem[\citeproctext]{ref-thecoca-colacompany2015}
The Coca-Cola Company. (2015). \emph{Coke Zero{\texttrademark} Tips Off Drinkable Advertising Campaign at NCAA® Men's Final Four® In Indianapolis}. Retrieved from \url{https://www.coca-colacompany.com/media-center/coke-zero-tips-off-advertising-campaign-at-ncaa-final-four}

\bibitem[\citeproctext]{ref-R-ggplot2}
Wickham, H. (2016). \emph{ggplot2: Elegant graphics for data analysis}. Springer-Verlag New York. Retrieved from \url{https://ggplot2.tidyverse.org}

\bibitem[\citeproctext]{ref-R-dplyr}
Wickham, H., François, R., Henry, L., Müller, K., \& Vaughan, D. (2023). \emph{Dplyr: A grammar of data manipulation}. Retrieved from \url{https://CRAN.R-project.org/package=dplyr}

\bibitem[\citeproctext]{ref-R-psych}
William Revelle. (2025). \emph{Psych: Procedures for psychological, psychometric, and personality research}. Evanston, Illinois: Northwestern University. Retrieved from \url{https://CRAN.R-project.org/package=psych}

\bibitem[\citeproctext]{ref-wu2005}
Wu, G. (2005). The Mediating Role of Perceived Interactivity in the Effect of Actual Interactivity on Attitude Toward the Website. \emph{Journal of Interactive Advertising}, \emph{5}(2), 29--39. \url{https://doi.org/10.1080/15252019.2005.10722099}

\bibitem[\citeproctext]{ref-R-knitr}
Xie, Y. (2015). \emph{Dynamic documents with {R} and knitr} (2nd ed.). Boca Raton, Florida: Chapman; Hall/CRC. Retrieved from \url{https://yihui.org/knitr/}

\bibitem[\citeproctext]{ref-yim2017}
Yim, M. Y.-C., Chu, S.-C., \& Sauer, P. L. (2017). Is Augmented Reality Technology an Effective Tool for E-commerce? An Interactivity and Vividness Perspective. \emph{Journal of Interactive Marketing}, \emph{39}(1), 89--103. \url{https://doi.org/10.1016/j.intmar.2017.04.001}

\end{CSLReferences}


\end{document}
